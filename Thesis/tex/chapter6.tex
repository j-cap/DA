\chapter{Summary and Outlook} \label{cha:summary}

The main goal of this thesis was to find a way of including a priori domain knowledege into the fitting process of a data-driven modeling approach. We showed that this is possible using the concept of shape-constraints, see~\pref{cha:solution-approach}, and B-splines as well as additive regression for multi-dimensional data as presented in~\pref{cha:practical-considerations} and~\pref{cha:practical-appl}. Further, we presented an overview of the related literature regarding the topics of linear models, model selection and B-splines in~\pref{cha:fundamentals}. 

With regards to future work, it would be interesting to try to create an algorithm that automatically finds the best possible combination of B-splines and tensor-product B-splines for a multi-dimensional problem with some a priori domain knowledge. For example, using 3 inputs, there are various combinations possible to create an additive model, e.g. a B-spline for dimension 1 and a tensor-product B-spline for dimension 2 and 3. An algorithm that automatically chooses the optimal combination with respect to some predefined criterion would help to enhance the usage of this approach. Another interesting aspect is to investigate further possible constraints that can be implemented using mapping and weighting matrices in the penalized least squares approach. An example of interest may be some kind of multi-peak constraint for tensor-product B-splines. 