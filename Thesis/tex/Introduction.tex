\chapter{Introduction}
Kapiteleinleitung [...]
Einstellungen im Textstudio lt. Bild \pref{fig:Bildlabel} und \pref{fig:Bildlabel2}. Biber muss installiert sein!
\begin{figure}
\includegraphics[width=\textwidth]{graphics/TexstudioEinstellungen.png}
\caption[Latex-Einstellungen im TextStudio - Bildunterschrift für Abbildungsverzeichnis.]{Latex-Einstellungen im TextStudio. Lange Bildunterschrift direkt unter dem Bild.}
\label{fig:Bildlabel}
\end{figure}

Sicherstellen, dass biber als Bibliographieprogramm in Texstudio hinterlegt ist.
\begin{figure}
	\includegraphics[width=\textwidth]{graphics/BiberAlsBibliographieprogramm.png}
	\caption{Biber als Standard-Bibliographieprogramm einstellen.}
	\label{fig:Bildlabel2}
\end{figure}

\section{Test rig}
A picture of the test setup for all experiments and measurements in this diploma thesis is shown in \pref{fig:testrig}. The motor (see parameters listed in \pref{tab:Motor} in \pref{apx:AppendixA}) with mass moment of inertia $\Theta_\text{m}$ is relatively stiff coupled with the linear bearing (see parameters listed in \pref{tab:Motor} in \pref{apx:AppendixA}) via an axial coupling set. In order to simulate different masses $m_\text{c}$ mounted at the moving cart, additional weights in form of iron blocks can be mounted on the cart, each with a mass of about \SI{1.3}{\kilogram}. The motor is used in closed-loop current controlled mode.

% Beispielbild mit tikz
\begin{figure}
	\begin{centering}
	\begin{tikzpicture}
	\node(image) at (0,0){\includegraphics[trim=0 0 0 0,clip, width=\textwidth,grid]{graphics/Testrig4.png}};
	\coordinate (basedrivingpulley) at (-5.5,-0.35);
	\node(pulleynote) at ($(basedrivingpulley)+(-0.5,0.8)$){$\Theta_\text{r}$};
	\draw(basedrivingpulley)--(pulleynote);
	\coordinate (baselinemotor) at (-6.52,-1.2);
	\coordinate (toplinemotor) at (-6.52,2.5);
	\coordinate (toplinecart) at (-3.68,3.5);
	\coordinate (baselinecart) at (-3.68,-0.75);
	\coordinate (pulley2base) at (6.85,-0.75);
	\coordinate (pulley2top) at (6.85,4);
	\coordinate (motortoptop) at (-6.52,4);
	\coordinate (pulley2mid) at(6.85,2);
	
	\coordinate (cartbaseleft) at (-4.38,-0.75);
	\coordinate (cartbaseright) at (-2.77,-0.75);
	 
	\draw (cartbaseleft)--++(0,2.25);
	\draw (cartbaseright)--++(0,3.25);
	\draw[latex-latex] ($(cartbaseleft)+(0,1.75)$)--($(toplinemotor)-(0,1.5)$)node[midway,above=0.1cm]{$l_1(\xi)$};
	
	\draw (baselinemotor)--($(motortoptop)+(0,0.5)$);
	\draw (baselinecart)--(toplinecart);
	\draw[latex-latex]($(toplinemotor)-(0,0.5)$)--($(toplinecart)-(0,1.5)$)node[midway, above=0.1cm,fill=white,opacity=\opac,text opacity=1.0]{$l_\xi$};
	\draw[-latex]($(toplinecart)-(0,0.5)$)--++(1.5,0)node[midway,above=0.1cm]{$\xi$};
	
%	\draw (toplinemotor)--(motortoptop);
	\draw (pulley2base)--($(pulley2top)+(0,0.5)$);
	\draw[latex-latex] (pulley2top)--(motortoptop)node[midway, above=0.1cm,fill=white,opacity=\opac,text opacity=1.0]{$l_3=\SI{79}{\centi\meter}$};
	\draw [latex-latex] (pulley2mid)--($(cartbaseright)+(0,2.75)$)node[midway, above=0.1cm, fill=white,opacity=\opac,text opacity=1.0]{$l_2(\xi)$};
	
	\node[fill=white,opacity=\opac,text opacity=1.0](mc) at (-3.7,0.8){$\mc$};
	\node[fill=white,opacity=\opac,text opacity=1.0](motor) at (-6.2,-1.28){motor $\Theta_\text{m}$};
	\node[fill=white,opacity=\opac,text opacity=1.0](Pulley2) at (6.91,-0.95){$\Theta_\text{r}$};
	\node[fill=white,opacity=\opac,text opacity=1.0] (cart) at (-3.7,-0.4){cart};
	\node[fill=white,opacity=\opac,text opacity=1.0] (axis) at (0,-1.05){linear bearing with toothed belt};
	
	\end{tikzpicture}
	\end{centering}
\caption[Picture of the test rig as used for experiments.]{Picture of the test rig for all practical experiments presented in this diploma thesis.}
\label{fig:testrig}
\end{figure}

\section{Literature survey}
Regarding the system identification process, Sch\"utte \cite{Schuette02} presents a framework for identifying belt driven servo mechanisms with test signals in a closed-loop approach. In contrast to \cite{Schuette02}, Henke \cite{Henke16} presents a pure model-based approach. .... 

\section{Structure of this thesis}
In \pref{cha:model}, the mathematical model of a belt driven servo mechanism is derived starting with a three-mass-spring-damper model. It is simplified to a parametric two-mass-spring-damper model...