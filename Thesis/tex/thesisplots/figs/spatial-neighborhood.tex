

\documentclass{article}
\usepackage[utf8]{inputenc}
\usepackage[english]{babel}

\usepackage{amsmath}
\usepackage{amsfonts}
\usepackage{amssymb}
\usepackage{graphicx}
\usepackage{enumerate}   
\usepackage{caption}
\usepackage{float}
\usepackage{dsfont}
\usepackage{csquotes}
\usepackage{tikz}
\usepackage{pgfplots}
\usepgfplotslibrary{groupplots}
\pgfplotsset{compat=1.16} 
\usepackage{array} 
\usepackage{booktabs}	% for horizontal lines
\usepackage{varwidth}% http://ctan.org/pkg/varwidth
\usepackage{csvsimple} % automatic table generation from csv files
\usepackage{comment}
\usepackage[numbers]{natbib}

%\bibliography{../../bibliography}


\begin{document}

\section{Introduction}

aasdasdasdasdasdasdasdasdasdasdasdasdasdasdasdasdasdasd
asdasdasdasdasdasdasdasdasdasdasdasdasdasd \cite{fahrmeir2007regression}.

\begin{comment}
	
\begin{figure}[H]
	\centering
	\begin{tikzpicture}
		\begin{axis}[
			xlabel = "x1",
			ylabel = "x",
			axis lines = none,
			ticks = none,
			scatter/classes={%
				e={mark=o},
				f={mark=*, mark options={fill=black, scale=2}},
				x={mark=x}}]
			\addplot[scatter,only marks,%
			scatter src=explicit symbolic]%
			table[meta=label] {
				x     y   label
				0.    0   e 
				0.2   0   e 
				0.4   0   e 
				0.6   0   e 
				0.8   0   e 

				0.    0.2  e 
				0.2   0.2  e 
				0.4   0.2  f 
				0.6   0.2  e 
				0.8   0.2  e 

				0.    0.4  e 
				0.2   0.4  f 
				0.4   0.4  x 
				0.6   0.4  f 
				0.8   0.4  e 

				0.    0.6  e 
				0.2   0.6  e 
				0.4   0.6  f 
				0.6   0.6  e 
				0.8   0.6  e 

				0.    0.8  e 
				0.2   0.8  e 
				0.4   0.8  e  
				0.6   0.8  e 
				0.8   0.8  e 
		};
	
	    % absolute in pgfplots coordinates
		\node[] at (axis cs: .055,.755) {$\beta_{1,1}$};
		\node[] at (axis cs: .45,.35) {$\beta_{j,r}$};
		\node[] at (axis cs: .475,.15) {$\beta_{j+1,r}$};
		\node[] at (axis cs: .475,.55) {$\beta_{j-1,r}$};
		\node[] at (axis cs: .275,.35) {$\beta_{j,r-1}$};
		\node[] at (axis cs: .675,.35) {$\beta_{j,r+1}$};	
		\node[] at (axis cs: .845, -0.05) {$\beta_{5,5}$};
\end{axis}
	\end{tikzpicture}
	\caption{Spatial neighborhood}
	\label{fig:sph}
\end{figure}
\end{comment}

\bibliographystyle{IEEEtranN}
\bibliography{../../bibliography}

\end{document}

