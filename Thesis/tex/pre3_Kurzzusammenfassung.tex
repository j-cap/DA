\addchap*{Kurzzusammenfassung}

Die massive Menge an Daten, die in industriellen Prozessen generiert wird, f\"uhrt zu einem zunehmenden Einsatz von datengetriebenen Methoden in Steuerungsanwendungen. A priori Dom\"anenwissen \"uber diese Prozesse ist oft vorhanden und wird nicht genutzt. Dennoch kann seine Einbeziehung die Qualit\"at und Robustheit des datengetriebenen Modells verbessern. 

In dieser Arbeit schlagen wir einen iterativen Algorithmus vor, um das gegebene a priori Dom\"anenwissen in den Modellierungsprozess unter Verwendung von shape-constraint P-Splines f\"ur ein- und zweidimensionale Probleme einzubeziehen. Der Algorithmus wird unter Verwendung des Konzepts der additiven Modelle auf h\"oherdimensionale Probleme erweitert. Wir evaluieren die Qualit\"at der vorgeschlagenen Methode anhand von Simulationsdaten und realen Beispielen.

Die Ergebnisse deuten darauf hin, dass die vorgeschlagene Methode die Qualit\"at der Modellierung insbesondere in Situationen mit sp\"arlichen und/oder verrauschten Daten verbessert. Wir kommen daher zu dem Schluss, dass die Verwendung von shape-constraint P-Splines zur Einbindung von a priori Dom\"anenwissen eine wertvolle Erweiterung f\"ur jedes datengetriebene Modellierungs-Toolset ist.

