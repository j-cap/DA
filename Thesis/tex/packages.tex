%%%%%%%%%%%%%%%%%%%%%%%%%%%%%%%%%%%%%%%%%%%%%%%%%%%%%%%%%%%%%%%%%%%%%%%%%%%%%%%%%%%%%%%
% added by j.weber 26.11.2020
\usepackage{multirow}
% added by j.weber 27.11.2020
\usepackage{mathtools}
\usepackage[ruled,vlined]{algorithm2e}
% added by j.weber 28.11.2020
\usepackage{comment}
% added by j.weber 03.12.2020
\usepackage{pgfplotstable}
% added by j.weber 27.01.2021
\usepackage{graphicx}
% added by j.weber 04.03.2021
%\usepackage{pdfpages}
%%%%%%%%%%%%%%%%%%%%%%%%%%%%%%%%%%%%%%%%%%%%%%%%%%%%%%%%%%%%%%%%%%%%%%%%%%%%%%%%%%%%%%%
%%%%%%%%%%%%%%%%%%%%%%%%%%%%%%%%%%%%%%%%%%%%%%%%%%%%%%%%%%%%%%%%%%%%%%%%%%%%%%%%%%%%%%%
\usepackage{acronym}
\usepackage{booktabs}
\usepackage{tikz}
%\usepackage{prettyref}

\usepackage{nomencl}
\usepackage[toc]{appendix}

\usepackage{pgfplots}
\usepgfplotslibrary{groupplots}
\usetikzlibrary{pgfplots.units}
\usetikzlibrary{shapes}
\usetikzlibrary{positioning}
\usetikzlibrary{calc}
\usetikzlibrary{arrows,scopes}
\usetikzlibrary{spy}
\usetikzlibrary{decorations.pathreplacing}
\usepackage{mathrsfs}

% for (i) (ii) as items
\RequirePackage{enumerate} 
% for [H] as figure placement
\RequirePackage{float}  

%\usepackage{showframe}%frames bei subcaption

\usepackage{siunitx}
\usepackage{nicefrac}

%include todonotes
\usepackage{todonotes}
% hack to make todonotes it work with externalize
\usepackage{letltxmacro}
\LetLtxMacro{\oldmissingfigure}{\missingfigure}
\renewcommand{\missingfigure}[2][]{\tikzexternaldisable\oldmissingfigure[{#1}]{#2}\tikzexternalenable}
\LetLtxMacro{\oldtodo}{\todo}
\renewcommand{\todo}[2][]{\tikzexternaldisable\oldtodo[#1]{#2}\tikzexternalenable}


\usepackage{xcolor}

\usepackage{calc}
\usepackage[labelformat=simple]{subcaption}
\renewcommand\thesubfigure{(\alph{subfigure})}

\pgfplotsset{compat=1.16,height=0.3\textheight,legend cell align=left,tick scale binop=\times}
\pgfplotsset{grid style={loosely dotted,color=darkgray!30!gray,line width=0.6pt},tick style={black,thin}}
\pgfplotsset{every axis plot/.append style={line width=0.8pt}}

\usepgfplotslibrary{external}
% Für die Verwendung von 'external' müssen die folgenden Anpassungen in Abhängigkeit der
% LaTeX Distribution durchgeführt werden:

% fuer Texlive: pdflatex.exe -shell-escape -synctex=1 -interaction=nonstopmode %.tex
%\tikzexternalize[shell escape=-shell-escape]   % fuer TeXLive

% fuer MikTeX:  pdflatex.exe -enable-write18 -synctex=1 -interaction=nonstopmode %.tex
\tikzexternalize[shell escape=-enable-write18] % fuer MikTex

%\pgfkeys{/pgf/images/include external/.code=\includegraphics{#1}} 
%\AtBeginDocument{
%	\@ifundefined{tikzexternalrealjob}{}{%
%		\message{*** Overriding the document specification for TikZ externalizer.}%
%		\ifthenelse{\equal{\jobname}{\tikzexternalrealjob}}{}{%
%			\gdef\maketitle{}%
%		}%
%	}%
%} 

% ordner zu tikz grafiken
\tikzsetexternalprefix{graphics/pgfplots/} % Ordner muss ev. zuerst haendisch erstellt werden
%\tikzset{external/up to date check=simple}

%\tikzset{external/system call={pdflatex \tikzexternalcheckshellescape --extra-mem-top=100000000 -halt-on-error-interaction=batchmode -jobname "\image" "\texsource"}} 

%Define command to get x and y values of a point
\newcommand{\gettikzxy}[3]{%
	\tikz@scan@one@point\pgfutil@firstofone#1\relax
	\edef#2{\the\pgf@x}%
	\edef#3{\the\pgf@y}%
}

\tikzset{external/system call= {pdflatex
		-extra-mem-top=5000000 
		-main-memory=9000000 
		\tikzexternalcheckshellescape 
		-halt-on-error 
		-interaction=batchmode
		-jobname "\image" "\texsource"}} 
	

%Force recompile	
%\tikzset{external/force remake}
\definecolor{amethyst}{rgb}{0.6, 0.4, 0.8}

\setlength{\nomitemsep}{-\parsep}

\makeatletter
\newenvironment{rrcases}{%
	\matrix@check\rrcases\env@rrcases
}{%
	\endarray\right\rbrace%
}
\def\env@rrcases{%
	\let\@ifnextchar\new@ifnextchar
	\left.
	\def\arraystretch{1.7}%
	\array{@{}l@{~}l@{}}%
}
\makeatother

\definecolor{limegreen}{RGB}{167, 229, 145}